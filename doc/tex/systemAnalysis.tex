\section{System Analysis}

%\subsection{System reach determination}

\subsection{System Requirements}

The requirements for the system are expressed here in a nested list way, each of
them with a textual and numeric reference so they are traceable. The functional
requirements are exposed first, followed by the other kinds of requisites needed
for the system.

\setlist[enumerate,2]{label*=\arabic*.}
\setlist[enumerate,3]{label*=\arabic*.}

\subsubsection{Functional Requirements}

\paragraph{Game Requirements}

\setlist[enumerate,1]{label=FRG \arabic*.}

\begin{enumerate}

	\item The state of the board can be shown to the user.
		\begin{enumerate}
			\item A text representation of each cell is printed.
				\begin{enumerate}
					\item A different character is used for each different state
						of a cell.
				\end{enumerate}
			\item The coordinates system is shown around the board.
				\begin{enumerate}
					\item Columns are shown as capital letters left to right
						starting with ``A'' and skipping ``I''.
					\item Rows are shown as numbers starting with 1 on the
						lowest row and increasing upwards.
				\end{enumerate}
		\end{enumerate}

	\item Movements can be introduced to be played on the board.
		\begin{enumerate}
			\item A move is introduced as the textual representation of the
				coordinates of the vertex to play on or as ``pass''.
				\begin{enumerate}
					\item The text introduced for the move must follow the
						regular expression \texttt{([A-Z][0-9]+|pass)}
					\item If the move is not valid, it must be notified to the
						user and another move asked for.
				\end{enumerate}
		\end{enumerate}

	\item The board will behave according to the Japanese rules of Go.

\end{enumerate}

\paragraph{Engine Requirements}

\setlist[enumerate,1]{label=FRE \arabic*.}

\begin{enumerate}

	\item Coordinates of the board representing valid moves must be printed.

\end{enumerate}

%\subsubsection{Security Requirements}
%
%\setlist[enumerate,1]{label=SR \arabic*.}

\subsubsection{Usability Requirements}

\setlist[enumerate,1]{label=UR \arabic*.}

\begin{enumerate}

	\item The engine executable will include a help option with the different
		modes of execution.

\end{enumerate}

\subsubsection{User Requirements}

\setlist[enumerate,1]{label=USR \arabic*.}

\begin{enumerate}

	\item For understanding the workings of the application the user needs to be
		familiar with the basics of the game of Go.

	\item For directly using the engine the user needs to be familiar with
		command line interfaces.

\end{enumerate}

\subsubsection{Technological Requirements}

\setlist[enumerate,1]{label=TR \arabic*.}

\begin{enumerate}

	\item The game program will be a python file able to be executed by the
		python interpreter.

	\item The program will make use of standard input and standard output for
		communication.
		\begin{enumerate}
			\item Standard input will be used for reading moves.
			\item Standard output will be used for showing the board and for
				messages directed to the user.
		\end{enumerate}

\end{enumerate}

\subsubsection{Response Time Requirements}

\setlist[enumerate,1]{label=RTR \arabic*.}

\begin{enumerate}

	\item The maximum thinking time of the engine will be configurable.
		\begin{enumerate}
			\item It will be possible to pass the maximum time as a launch
				argument.
			\item It will be possible to store the maximum time as a setting
				in a configuration file
		\end{enumerate}

\end{enumerate}


\setlist[enumerate,1]{label=\arabic*.}

\subsection{System Actors}

There are various actors who will interact with the system, both human and
non-human.

\begin{itemize}

	\item The human player who interacts with the playing interface.
	\item The human user who interacts with the engine.
	\item A GUI software which uses the engine to generate moves.

\end{itemize}

\subsection{Use Cases}

\begin{figure}[h]
	\begin{center}
		\includegraphics[width=\textwidth]{diagrams/useCases.png}
		\caption{Use cases.}\label{fig:useCases}
	\end{center}
\end{figure}

The different actors and use cases are represented on \fref{fig:useCases}. Each
use case is explained next.

\paragraph{Play a match}

The game interface reads the moves presented by the player and shows their
result on the board.

\paragraph{Generate moves}

The engine interface reads the input for generating a move as stated by the
GTP protocol and outputs the coordinates of the board to play.

\paragraph{Use as backend for machine player}

The engine is used as the backend for generating moves for a machine player.

\subsection{Subsystems}

\subsubsection{Subsystems description}

There will be two main subsystems.

% TODO: Are there really two different subsystems? They look very coupled, since
% the engine will use some classes of the game. This section is more suited for
% independently run systems which communicate through some common interface.

The first, called the Game System, will be in charge of storing all the state
information regarding a Go match, such as the history of moves, the possible
variations, the state of the board at any given time or the current number of
captured stones.

The second, called the Engine System, will implement the GTP interface and use
the Game System to analyze positions and generate moves via decision algorithms.

\subsection{Class analysis}

The classes resulting from the analysis phase are shown in
\fref{fig:analysisClasses}.

\begin{figure}[h]
	\begin{center}
		\includegraphics[width=\textwidth]{diagrams/analysisClasses.png}
		\caption{General classes obtained from the analysis
		phase.}\label{fig:analysisClasses}
	\end{center}
\end{figure}
