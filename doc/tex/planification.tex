\section{Planning}

\subsection{Driving needs}

As one of the deepest and most studied games in the world, Go presents a very
interesting problem for artificial intelligence. Implementing not only the
game's simple but subtle rules, but a system capable of playing it with a
satisfying level of skill, is a task worth of pursuing as an exercise on
software design, algorithmics and AI research.

On the practical level, this project can be a foundation for the development of
different Go analysis algorithms by providing an existing engine to house them,
which can be of interest to Go players and software scientists alike.

\subsection{Reach}

Presented here are the ideal targets of the project.

\begin{itemize}
	\item An implementation of the game of Go, that is, a system for holding the
		moves and variants of a match (a tree of moves) and the logic for the
		game's rules.
	\item An engine capable of analyzing board positions and generating strong
		moves.
	\item Either a GUI specifically developed for the project or an
		implementation of an existing protocol so the engine can be used with
		existing tools and GUIs.
\end{itemize}

\subsection{Project stages}

The project will be organized in several stages based on the different
components and needs.

\subsubsection{Game implementation}

The rules of the game must be implemented, ideally in a way they can be tested
by direct human play. This system will at its bare minimum represent the
Japanese Go rules (area scoring, no superko rule, no suicide moves).

\subsubsection{Engine implementation}

The key of this project is to create some kind of system able to generate strong
moves based on any given board configuration: this will be such system. It will
implement an existing protocol so it can be used with other compatible tools. It
has to be able to receive game updates and configuration and to output moves for
either player. It should also be modular enough so different algorithms can be
selected and tested against each other as an experimental search for the best of
them.

\subsubsection{Artificial Intelligence algorithms}

Different algorithms for the engine to use should be implemented and tested. The
results of this development and testing process should be presented as part of
the final version of the project.

\subsection{Logistics}

The project will be developed by Íñigo Gutiérrez Fernández, student of the
Computer Software Engineering at the University of Oviedo, with supervision from
Vicente García Díaz, Associate Professor in the Department of Computer Science
at the University of Oviedo.

The used material consists of a development and testing machine owned by the
student with specifications stated later on the project duration.

\subsection{Work plan}

The sole developer will be the student, who is currently working as a Junior
Software Engineer on a 35 hour per week schedule and no university
responsibilities other than this project. Taking this into account, a sensible
initial assumption is that he will be able to work 3 hours a day, Monday to
Saturday. Gantt diagrams for the planned working schedule are shown as
Fig.~\ref{fig:planificationWorkPlanGame} and
Fig.~\ref{fig:planificationWorkPlanEngine}.

\begin{figure}[h]
	\begin{center}
		\includegraphics[width=\textwidth]{diagrams/planificationWorkPlanGame.png}
		\caption{Initial work plan for implementing the game.
		}\label{fig:planificationWorkPlanGame}
	\end{center}
\end{figure}

\begin{figure}[h]
	\begin{center}
		\includegraphics[width=\textwidth]{diagrams/planificationWorkPlanEngine.png}
		\caption{Initial work plan for implementing the engine and algorithms.
		}\label{fig:planificationWorkPlanEngine}
	\end{center}
\end{figure}

\subsection{Previous works}

\subsubsection{Existing engines}

\paragraph{AlphaGo}

AlphaGo is a Go play and analysis engine developed by DeepMind Technologies, a
company owned by Google. It revolutionized the world of Go in 2015 and 2016 when
it respectively became the first AI to win against a professional Go player and
then won against Lee Sedol, a Korean player of the highest professional rank and
one of the strongest players in the world at the time. Its source code is
closed, but a paper \parencite{natureAlphaGo2016} written by the team and
published on Nature is available on
https://storage.googleapis.com/deepmind-media/alphago/AlphaGoNaturePaper.pdf.

The unprecedented success of AlphaGo served as inspiration for many AI projects,
including this one.

\paragraph{KataGo}

KataGo is an open source project based on the AlphaGo paper that also achieved
superhuman strength of play. The availability of its implementation and
documentation presents a great resource for this project.

\paragraph{GnuGo}

GnuGo is a software capable of playing Go part of the GNU project. Although not
a strong engine anymore, it is interesting for historic reasons as the free
software engine for which the GTP protocol was first defined.

\subsubsection{GTP}

GTP (\textit{Go Text Protocol}) is a text based protocol for communication with
computer go programs. [ref https://www.lysator.liu.se/~gunnar/gtp/] It is the
protocol used by GNU Go and the more modern and powerful KataGo. By supporting
GTP the engine developed for this project can be used with existing GUIs and
other programs, making it easier to use it with the tools users are already
familiar with.

\subsubsection{Sabaki}

Sabaki is a go board software compatible with GTP engines. It can serve as a GUI
for the engine developed in this project and as an example of the advantages of
following a standardized protocol.

\subsection{Technological Infrastructure}

\subsubsection{Programming language}

The resulting product of this project will be one or more pieces of software
able to be run locally on a personal computer. The programming language of
choice is Python, for various reasons:

\begin{itemize}

	\item Has established a reputation on scientific fields and more
		specifically on AI research and development.
	\item Interpreters are available for many platforms, which allows the most
		people possible to access the product.
	\item Although not too deeply, it has been used by the developer student
		during its degree including in AI and game theory contexts.

\end{itemize}

\subsubsection{Interface}

Both the game and the engine will offer a text interface. For the game this
allows for quick human testing. For the engine it is mandated by the engine,
since GTP is a text based protocol for programs using text interfaces.
Independent programs compatible with this interface can be used as a GUI.
